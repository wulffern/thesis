\section{Figure of Merit}
Figure of merit is one of the measures used to compare ADCs with
eachother. A figure of merit tries to incorperate some of the
important parameters for an ADC. One figure of merit which is commonly
used compares power dissipation , sampling frequency and effective
number of bits. It is usually given as
\begin{equation}
  \label{eq:fomwords}
  FOM = \frac{P}{2^{2ENOB} f_s}
\end{equation}


\subsection{FOM for constant current }
We have previously defined the necessary sampling capacitance as 
\begin{equation}
  \label{eq:csamp}
   C_s = \frac{ \eta_1 12 kT 2^{2ENOB}}{V{pp}^2}
\end{equation}
where $\eta_1$ is the scaling factor between the thermal noise power
and quantization noise power, a usual number is $\vnn{th}/\vnn{q} = 1/4$.


\subsection{FOM for constant unity gain}
The output voltage $V_o$ for linear settling with a single pole is
given by 
\eqna{
  \label{eq:epsilon}
  V_o &=& V_i(1 - e^{-t/\tau})   \\
 V_i - \epsilon &=& V_i - e^{-t/\tau}\\
 \epsilong = e^{-t/\tau}
}
Assume two period switched capacitor circuit, here $t$ will be given
by $1/2f_s$. The time constant $\tau$ can be defined as 
\eqn{
  1/\tau_u = \omega_u = g_m/C_s = 2I_D/V_{EFF}C_s
}
Epsilon is the linear settling error and must be smaller than $ 1/2^b $
to avoid a settling error, we thus define $\epsilon =   1/2^b$.

Equation \req{epsilon} can, if we insert for the known variables $C_s,
t, \tau$, be written

\eqn{
  1/2^b = e^{\frac{-I_D}{V_{EFF} \frac{ \eta_1 12 kT
      2^{2ENOB}}{V{pp}^2} f_s}}
}

If we insert define $V_{EFF} = \eta_2 V_{DD}$ and $V_{PP} = \eta_3
V_{DD}$ and rewrite

\eqn{
 ln (1/2^b) = \frac{ - \eta_3^2 P  }{\eta_2 \eta_1 12
   kT 2^{2ENOB}f_s}
}

Which results in 

\eqn{
FOM = \frac{ \eta_2 \eta_1 12 k T ln (1/2^b)}{\eta_3^2}
}

The constants $\eta_3$ and $\eta_2$ have a  maximum value of 1, but it
is unusuall for $\eta_2$ be bigger than 1/4. Assume $\eta_ 
%%% Local Variables: 
%%% mode: latex
%%% TeX-master: "thesis"
%%% End: 
