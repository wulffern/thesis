%%%%%%%%%%%%%%%%%%%%%%%%%%%%%%%%%%%%%%%%%%%%%%%%%%%%%%%%%%%%%%%%%%%%%%
%% Fpreface.tex
%% Description:   
%% Author:        Carsten Wulff <wulff@iet.ntnu.no>
%% Created at:    Thu Apr 24 16:35:47 2008
%% Modified at:   Mon Nov  3 17:18:51 2008
%% Modified by:   Carsten Wulff <wulff@iet.ntnu.no>
%%%%%%%%%%%%%%%%%%%%%%%%%%%%%%%%%%%%%%%%%%%%%%%%%%%%%%%%%%%%%%%%%%%%%%

\chapter*{Preface}\addcontentsline{toc}{chapter}{Preface}
This thesis was submitted to the Norwegian University of Science and Technology 
(NTNU) in partial fulfilment of the requirements for the degree of philosophiae 
doctor (PhD). 
The work presented herein was conducted at the Department Electronics
and Telecommunication, NTNU, under the supervision of Professor Trond
Ytterdal, with Professor Trond S{\ae}ther as co-advisor. Financial support from the Norwegian Research Council through the
project Smart Microsystems for Diagnostic Imaging in Medicine (project
number 159559/130) and the project ASICs for Microsystems (project
number 133952/420) is gratefully acknowledged.


\section*{Research Path}
This is a document that has been four years in the making. I began by
my  work in January 2004. The intent was to investigate calibration
algorithms for micro-systems, with focus on genetic algorithms. But I
strayed from this path and found analog-to-digital converters. The project
Smart Microsystems for Diagnostic Imaging in Medicine (SMIDA) needed a
low resolution high speed ADC, and I was asked to build it. This led to some initial work on dynamic comparators, opamps in
90nm CMOS and bootstrapped switches. 

Wislands doctoral thesis (2003) on \textit{Non-feedback Delta-Sigma
  modulators for digital-to-analog conversion} peaked my
interest. We\footnote{ Trond Ytterdal and I}
wanted to see if we could apply the open-loop sigma-delta technique to
analog-to-digital converters. We believed they could be used as
front-ends to pipelined ADCs. In that respect, we developed techniques
for switched-capacitor circuits. 

At ISSCC 2006 the first comparator-based switched-capacitor circuit
was published, and we immediately jumped on it. From the summer of 2006
to the summer of 2007 my time was dedicated to tape-out the first
differential comparator-based switched-capacitor ADC. That year
I was fortunate to spend my time at the
University of Toronto as a visiting researcher.  The time in Toronto inspired much
of my work, like the open-loop residue amplifiers for pipelined ADC,
and the continuous time bootstrapped switches. 

My chip came back in January 2008, and most of the spring was spent on
making the chip work. On the first day I
got 4.2-bit ENOB, and it took me four months to get to 7.05-bit.

As these four and a half years draw to a close, I find that I am
satisfied. In a sense I have come full circle with the genetic
algorithm used to calibrate my ADC. 


\section*{Acknowledgements}
\begin{itemize}
\item To my wife, Anita, without her this thesis
would not have seen the light of day. She was willing to move half way
around the world with me, for which I am forever grateful. 
\item To my son, Villem, born during my years as a PhD student you have shown me that
  it's possible to get up at 05:30 {\sc a.m} and still
  survive the day. This has been immensely helpful. Your smiles and hugs
  brighten my mood after a long day at work.
\item To my late mother, my dad, my sisters and my extended family for
  your love and support.
\item To my supervisor, Trond Ytterdal, he has always been available
  for questions and his guidance is valued. He has provided the resources necessary to do this
  work.  
\item To my co-advisor, Trond S{\ae}ther, for his support and
  convincing me that analog integrated circuits was the way to go.
\item To my colleagues at Department of Electronics and
  Telecommunication, Circuits and Systems Group for lunch discussions,
  coffee sessions and support.
\item To my fellow PhD students for many
  fun discussions: {\O}ystein
  Gjermundnes, Ashgar Havashki, Are Hellandsvik, Tajeshwar Singh, Ivar
  L{\o}kken, Anders Vinje, Saeeid Tahmasbi Oskuii,
   Linga Reddy Chenkeramaddi and Bertil Nistad. 
\item To Johnny Bj{\o}rnsen for help with measurements.
\item To my fellow graduate students during my stay in Toronto for
  their hints and tips: Imran Ahmed, Bert Leesti, Ahmed Gharbiya,
  Trevor Caldwell and Pradip Thachile.
\item  A special thanks to Professors
  Ken Martin and David Johns for many useful questions and
  suggestions.  
\end{itemize}

\section*{Comments on style}
In a break from conventional page numbering this thesis start with
page number 1 on the title page. In this digital age it is likely that this
thesis will be read on a computer. As the title page is page one, the
page numbers of the thesis will match the page numbers of the electronic
document. I believe this will make the thesis easier to navigate.


