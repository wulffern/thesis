%%%%%%%%%%%%%%%%%%%%%%%%%%%%%%%%%%%%%%%%%%%%%%%%%%%%%%%%%%%%%%%%%%%%%%
%% Fsc-approach.tex
%% Description:   
%% Author:        Carsten Wulff <wulff@iet.ntnu.no>
%% Created at:    Fri Sep 14 11:09:59 2007
%% Modified at:   Fri Jun 13 18:19:28 2008
%% Modified by:   Carsten Wulff <wulff@iet.ntnu.no>
%%%%%%%%%%%%%%%%%%%%%%%%%%%%%%%%%%%%%%%%%%%%%%%%%%%%%%%%%%%%%%%%%%%%%%

\section{Approaches to Switched Capacitor Circuits}


It is interesting to investigate how difficult it is to achieve a
certain DC gain in different technologies. If we restrict our search to
three types of OTAs--- differential pair, folded cascode, and folded
cascode with gain boosting ---, and different permutations of these, we can give an estimate based on the
intrinsic gain--- given by the transconductance over the output conductance
--- of a technology. 

\subsubsection{OTA DC gain estimates based on architecture}

\paragraph{Folded Cascode (FC)}
The FC is a decades old workhorse OTA that is very common in switched
capacitor circuits. There are many variations on the FC, but the basic circuit
can be seen in Figure \label{fig:foldedcasc}. The input differential
pair provide the transconductance, and the cascode serves to increase the
output resistance. If we assume that the transconductance of the input
transistors is equal to that of the cascode transconductance, and that
the intrinsic gain is defined as $G_i = g_m/g_{ds}$, we can
approximate the DC gain by
\eqn{
  A_{FC} = g_{mi} \times g_{m} r_{ds}^2/2 = \frac{1}{2} G_i^2
}

\paragraph{Common Source (CS)}
The DC gain of a common source amplifier can be approximated by
\eqn{
  A_{CS} = \frac{1}{2} G_i
}

\paragraph{Two Stage OTA}
If we make a two-stage amplifier the total DC gain is a multiplication of
the individual gains.
\eqn{
A_{tot} = A_{FC}A_{CS} = \frac{1}{4}G_i^3
}

\paragraph{Gain Boosting}
Gain boosting is common in folded cascode OTA to further increase the
output resistance. The technique generally increases the output
resistance proportional to the gain of the gain boosting
amplifier. Ergo the gain of a folded cascode with folded cascode gain
boosting is
\eqn{
A_{FC[FC]} =
A_{FC}A_{FC}
= \frac{1}{4} G_i^4
}


Table \label{tab:otagain} gives an overview of the estimated DC gain for
different configurations of OTA. The first four OTAs  are well seasoned
OTAs and can be found in many publications \cite{berntsen05}, \cite{ahmed05}. The last
two will not be trivial to design and they will dissipate large amounts of
power. The last two OTAs might be so difficult to design that it is well
worth an effort to try to find alternate approaches.

\begin{table}[ht]
\caption{Gain estimates for different OTA architectures}
\centering 
\begin{tabular}{ l | l | c }

\label{tab:otagain}
Gain Name&Architecture & Gain Estimate\\
\hline 
$A_{CS}$&CS & $\propto G_i$\\\hline 
$A_{FC}$&FC & $\propto G_i^2$\\\hline 
$A_{FC-CS}$&FC with CS second stage & $\propto G_i^3$\\\hline 
$A_{FC[FC]}$&FC with FC gain boosters & $\propto G_i^4$\\\hline 
$A_{FC[FC]-CS}$&FC with FC gain boosters and CS second stage & $\propto G_i^5$\\\hline 
$A_{FC[FC]-FC}$&FC with FC gain boosters and FC second stage & $\propto G_i^6$\\
\end{tabular}
\end{table}


\subsubsection{OTA gain in different technologies}

In the recent CMOS technology nodes we have seen a drastic reduction in
intrinsic gain, making it harder and harder to design high DC gain
OTAs. In 65nm we have reached a point were it might not be practical
to design OTAs with 100dB gain. Such an OTA will be a severe challenge
to design, it will probably have to be a three stage OTA, or a two
stages with gain boosting. Some of the challenges in the design of
such an OTA include ensuring stability, achieving low power
dissipation and designing multiple stable common mode
feedback loops.

The intrinsic gain can be used to estimate the DC gain of
OTAs. Table \label{tab:ingain} shows the intrinsic gain for
transistors in different technology nodes. We've used a width to
length ratio of ten and a length of 1.2 times the minimum length. All
transistors were biased at an effective gate overdrive of $V_{EFF} = V_{GS} - V_{TH} =
V_{DD}/8$ and a drain to source voltage of $V_{DS} =
V_{DD}/2$. Table \label{tab:ingain} shows the gain estimates in dB for different
OTA architectures. In 350nm and 180nm  more than $100dB$ gain is feasible without
too much difficulty. But in 90nm and 65nm it would be very challenging
to design an OTA with more than $100dB$ gain. The
values in Table \label{tab:ingain} are what you get for free in each
technology, any more gain for a given topology will cost time and
effort. Table \label{tab:realota} shows the gains of
some published OTAs.

\begin{table}[ht]
\caption{OTA gain estimates for different technology nodes and OTA architectures. Simulations in SPICE using $W/L=10, V_{EFF}=V_{DD}/8, V_{DS}
  = V_{DD}/2, L=1.2L_{min}$. All gains are in dB.}
\centering 
\begin{tabular}{ l | c | c | c | c | c | c | c }

\label{tab:ingain}
Tech. node&$G_i$&$A_{CS}$&$A_{FC}$&$A_{FC-CS}$&$A_{FC[FC]}$&$A_{FC[FC]-CS}$&$A_{FC[FC]-FC}$\\
\hline 
65nm	&9	&13     &32	&45	&64	&77	&96\\
90nm	&13	&16     &39	&55	&77	&93	&116\\
130nm	&17	&19     &43	&62	&86	&105	&130\\
180nm	&26	&22     &51	&73	&101	&123	&152\\
350nm	&60	&30     &65	&95	&130	&160	&195\\
\end{tabular}

\end{table}

\begin{table}[ht]
\caption{DC gain of published OTAs}
\centering 
\begin{tabular}{ l | l | l | r  }
\label{tab:realota}
Reference & Architecture & Technology & Gain\\
\hline 
\cite{berntsen05} & FC with FC gain boosters & 90nm & 70dB\\
\cite{berntsen05} & FC & 90nm & 30dB\\
\cite{ahmed05} & FC with FC gain boosters & 180nm & 105dB\\
\cite{hernes07} & CS & 130nm & 25dB\\
\end{tabular}
\end{table}



\subsection{Future of OTA based SC Circuits}
OTA based SC circuits will probably be with us for a long time. In
older technology nodes it will remain the preferred method of doing SC
circuits, mostly because it is a well researched area, and there is
ample information on how to do good OTA based SC circuits. In 90nm
CMOS and below, however, other implemetations than straight OTA based
SC circuits will probably have to be used. As of yet there is no
obvious path forward, two of the approaches are: living with the low
DC gain, and changing the SC implementation completely. 

\subsubsection{Living with low DC gain}
There has been a surge of publications where some form of digital
calibration is used to compensate for low DC gain. Some of the interesting
publications in this area are the statistics based gain calibration of
Siragusa \& Galton \cite{siragusa04}. And ``Split-ADC'' calibration
technique of McNiell, Coln \& Larviee \cite{mcniell05} and Li \& Moon
in \cite{li03}. More recently from Hernes et al. \cite{hernes07} a
 calibrated high performance, high speed ADC which uses single stage
 OTAs with a DC gain of 25dB. This is an interesting approach to the
 gain problem in CMOS technology, but it will not be discussed further
 in this thesis.

\subsection{Alternative approach to SC circuits}
Two alternatives to switched capacitor circuits have been investigated
in this thesis work
\begin{enumerate}
\item Comparator-Based Switched-Capacitor (CBSC) Circuits 
\item Open Loop Amplifier Based Switched Capacitor Circuits
\end{enumerate}


%---------------------------------------------------------------------
\section{Comparator based SC Circuits}
%---------------------------------------------------------------------
At the time of writing there were two proven silicon CBSC ADCs, the first by Sepke
\cite{sepke06} in 2006 and the second by Brooks \cite{brooks07} in
2007. One publication based on simulation
results of a sigma-delta ADC by Prelog \cite{prelog07} was published in 2007. An extended paper of
\cite{sepke06} was published by Fiorenza in \cite{fiorenza06}. In
Sepke's thesis\cite{sepke06th} there was a thorough analysis of noise
in CBSC circuits, which will not be discussed in any detail in this
thesis. 

The output value of a switched-capacitor circuit needs only be correct
during a certain time of a clock period, when the next stage samples
the output. How the SC circuit arrives at this value is of no
consequence. In \cite{sepke06} they introduced the CBSC technique
which is a completely different approach to SC circuits. Instead of
forcing the virtual ground condition, where we have $V_x=0$ in
Figure \label{fig:sc-ota}, the output is charged using a current
source and a comparator detects when $V_x = 0$ as shown in Figure\label{fig:sc-cbsc}.
\myfigure{graphics/sc-ota}{OTA-based switched-capacitor circuit}{sc-ota}
\myfigure{graphics/sc-cbsc}{Comparator-based switched-capacitor circuit}{sc-cbsc}



%%% Local Variables: 
%%% mode: latex
%%% TeX-master: "thesis"
%%% End: 
