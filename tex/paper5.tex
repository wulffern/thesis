\chapter{Paper 5}\label{sc:p5}
\textbf{\Large Design of a 7-bit 200MS/s, 2mW Pipelined ADC With Switched
  Open-Loop Amplifiers In a 65nm CMOS Technology}\\
\indent Carsten Wulff and Trond Ytterdal\\
\indent In proceedings of the 25th NORCHIP Conference, 2007.\\
\indent  Digital Object Identifier 10.1109/NORCHP.2007.4481042 \\
\renewcommand\myfigname{p2fig:}
\renewcommand\myeqname{p2eq:}
%\section{Abstract}

\begin{abstract}
We present the design of a
7-bit 200MS/s pipelined ADC with switched open-loop amplifiers in a
65nm CMOS technology. As a
result of turning
off the open-loop amplifiers during sampling we reduce the power
dissipation by 23\%. The ADC achieves a SNDR
of 40dB close to the Nyquist frequency, with a power dissipation of
2mW, which results in a 
Walden FOM of 0.13pJ/step and a Thermal FOM of 1.6fJ/step.
\end{abstract}


%-------------------------------------------------------
\section{Introduction}
%-------------------------------------------------------
Low resolution high speed ADCs have historically been
Flash based architectures. The Flash ADC is a fast architecture due to
it's parallel nature. However, it is a very inefficient
architecture. One of the
reasons for its inefficiency is that it is not possible, with current
processing technology, to get the input capacitance down to what is
the minimum required by thermal noise. This is mostly due to parasitic
capacitances from transistors and metal routing. 

Before we begin our argumentation we should define what we mean with
parasitic capacitances. We define parasitic capacitance as; any
capacitance that is not required by the operation of the circuit.

% For
%example in a standard sample and hold the only required capacitance in our
%circuit would be the sampling capacitance, as shown by Figure
%\ref{p2fig:samplehold}. 
%Parasitic capacitances at the sampling node
%include the transistor capacitances (Cgd Cdb), sampling
%capacitor parasitics (Cp), input capacitance of next stage (Cin) and routing
%capacitances (Cmet).

%\myfigure{samplehold}{ Sample and hold with buffer, parasitics are
%  represented by gray capacitances.}{samplehold}

One of the fundamental error sources that limit the performance of ADCs is the thermal noise power,
which is usually represented as
%\eqn{
%\label{p2eq:thermpower}
%$\overline{V_{thermal}^2(t)} = \frac{a\times k T}{C}$
%}
$\overline{V_{thermal}^2} = a\times k T/C$
where a is a constant, k is Boltzmann's constant, T is the temperature
in Kelvin and C is the sampling capacitance. 

%Thermal noise also manifests
%through sampling jitter, but in a 7 bit ADC at 200Msps jitter is not the
%dominating error source. \footnote{Jitter in an ADC is a sum of the
%  source jitter from the clock and the clock phase generating circuit
%  in the ADC. In the clock phase generating circuit the jitter with a white power spectral density is usually caused by
%  thermal noise}

Thermal noise power sets the lower limit for how much power we must
dissipate to achieve a certain SNR. With respect to thermal noise
power a certain SNR usually translates into a certain sampling
capacitance. 

%If we assume the least significant bit (LSB) is given by
%\eqn{
%V_{LSB} = \frac{V_{max}}{2^{bits}}
%}
%which leads to a quantization noise power of
If we assume a quantization noise power of
%\eqn{
%$\overline{V_{LSB}^2(t)} = \frac{V_{LSB}^2}{12}= \frac{V_{max}^2}{2^{2bits}\times
%  12}$
$\overline{V_{LSB}^2} = V_{LSB}^2/12= V_{max}^2/(2^{2bits}\times
  12)$
%}
, and we assume $\frac{1}{4}\overline{V_{LSB}^2} = \overline{V_{thermal}^2}$ and
$a = 1$, the equation for the required sampling capacitor is
\eqn{
\label{p2eq:capval}
C = \frac{48 k T 2^{2bits}}{V_{max}^2}
}

If we use 6 bits, $k=1.38\times10^{-23} J/K$, $T=353 K$ and $V_{max} = 0.4V$ 
the required sampling capacitance is 6fF. This, of course, assumes that other concerns like
mismatch or unwanted effects from parasitic capacitances does not come
in to play.

%In an ideal world this would be the largest capacitor in our ADC, and
%the power dissipation would be proportional to this capacitor. However, in current nanoscale technology it is not uncommon to get in
%excess of 10fF parasitic capacitance at circuit nodes. 
%So a 6fF sampling capacitance is smaller than the parasitic
%capacitance. Because of this, we will use more power to charge and
%discharge parasitic capacitances than the needed sampling capacitance,
%through this we waste power. 

%And often it will not be possible to have
%a sampling capacitor of 6fF because the parasitics will affect the
%circuit operation, which is the case with the pipelined ADC we present
%in this paper. In pipelined ADC with open-loop amplifiers, the input
%capacitance of the amplifier  decreases the gain of the
%multiplying DAC, as explained in \cite{Shen07}.

Parasitics at a node can reach 10fF in current nanoscale
technologies, as a consequence the parasitics can be larger than required
sampling capacitor at the 6-bit level. This is
one reason why a Flash ADC looses when it comes to
efficiency. A 6-bit Flash ADC without averaging or interpolation has 64 comparators connected
to the input, each of which has possibly 10fF input capacitance. Thus
a Flash ADC can have 600fF input capacitance, which is two orders of magnitude larger than the
required sampling capacitance. 

If we look at the figure of merit (FOM) of 6-bit ADCs, and higher
resolution ADCs, using the Walden FOM \cite{walden99} given by
\eqn{
FOM = \frac{P_{diss}}{2^{bits}f_s}
}
where $P_{diss}$ is the
power dissipation and $f_s$ is the sampling frequency. And the Thermal
FOM given by 
\eqn{
\label{p2eq:thermfom}
FOM = \frac{P_{diss}}{2^{2bits}f_s}
}
We get  Figure \ref{p2fig:waldenthermal}, with the gray
diamonds representing Walden FOM and the black triangles representing
Thermal FOM. The data for Figure
\ref{p2fig:waldenthermal} was collected from ADCs published in the Journal of
Solid State Circuits in the years 1975-2005. According to the Walden FOM there
are 6-bit ADCs that are just as good as the 15-bit ADCs, since they
have the same figure of merit. However, the Walden FOM is an
empirically deduced FOM, and it has come
under some scrutiny in the recent years. A more theoretically
correct FOM is the Thermal FOM.\footnote{As far as we
  know the FOM has not yet been given a name, so the name Thermal FOM
  is not an official name. It is, however, a descriptive name.}

\myfigure{graphics/waldenthermal}{ Walden FOM versus Thermal FOM as a function
  of bits for ADCs published in the
  Journal of Solid State Circuits 1975-2005. Thermal FOM in black and
  Walden FOM in gray.}{waldenthermal}
%\myfigure{thermal}{ Thermal FOM as a function of bits for ADCs in
%  Journal of Solid State Circuits 1975-2005}{thermal}

The argument for why the Thermal FOM is more correct is as follows. Assume a thermal noise limited ADC, where
the power dissipation is proportional to the sampling
capacitance. 
If we increase the resolution by one bit, we can see from
(\ref{p2eq:capval}) that the sampling capacitance quadruples, and
through this the power dissipation quadruples.
However, the Walden FOM only allows a doubling of the
power dissipation for each bit of resolution added. This leads to an
unfair FOM for high number of bits and a lenient FOM for low
number of bits. 

%If we re-plot the data from Figure \ref{p2fig:waldenthermal}
%with the FOM given by \ref{p2eq:thermfom} we get Figure
%\ref{p2fig:thermalthermal}. 


%Most of this difference is because low
%resolution ADCs are not close to being limited by thermal noise.

Two alternatives to Flash ADCs have received some attention in recent years. The first is
successive-approximation (SAR) ADCs and the second is pipelined ADCs. Both
have in recent papers achieved good results. 

%In \cite{draxelmayr04}
%they achieved a Thermal FOM of 20fJ/step with a time interleaved SAR
%ADC. 
In \cite{Chen06} they presented a
6-bit 600MS/s time interleaved asynchronous successive-approximation
ADC, with a Walden FOM of 0.22pJ/step and a Thermal FOM of
5.7fJ/step. In \cite{Shen07} they presented a 6-bit pipelined ADC with open-loop
amplifiers achieving a Thermal FOM of 59fJ/step.
Note that the best 6-bit ADC in Figure \ref{p2fig:waldenthermal}, is 
\cite{lin02} with a Thermal FOM of 16.38fJ/step, which is three orders of magnitude worse than
the best $>$ 14-bit ADCs. And the best 7-bit ADC is a 100kS/s SAR
\cite{Scott03} with a Thermal FOM of 1.89fJ/step, which is two orders of magnitude worse than the best
$>$ 14 bit ADCs.
% None of these get
%close to the apparent Thermal FOM limit of $0.01fJ/step$ in Figure \ref{p2fig:waldenthermal}, achieved by some of the
%higher resolution converters. 
%It is probably not possible to get close
%to this limit in current
%technologies, but it is interesting to see how close one can get at
%reasonably high speed. 

Our goal for this design was to optimize the FOM for a low resolution
ADC at a resonable speed. 
We choose to design a pipelined ADC and placed it
at the conservative speed of 200MHz. That is, the speed is
conservative if we compare to the speed to
\cite{Shen07} or \cite{Chen06}. Usually, low resolution ADCs are used
in the GHz range, so we assume that the this pipelined ADC would be
used in a time interleaved architecture. Knowing that we would be unable
to use a 6fF sampling capacitor we opted for 7-bit resolution, since
the thermal noise power would anyway be low because of the larger
sampling capacitor, and adding one more stage
does not increase the power significantly. To keep parasitic
capacitances from transistors as low as possible we chose to use a
65nm CMOS technology. The pipelined ADC architecture is
explained in the following section with results of simulations
presented in Section \ref{p2results}.


%------------------------------------------------------------------------
\section{Pipelined Architecture}\label{p2arch}
%------------------------------------------------------------------------
The ADC was designed in a 65nm low power CMOS technology with low threshold
voltage (lvt) transistors. 
The architecture of the differential pipelined ADC is shown in Figure \ref{p2fig:pipe}. The ADC has
five 1.5-bit pipelined stages and a three level flash at the end. Each stage has a three level analog to
digital converter (SADC) and a multiplying DAC (MDAC). The MDAC has a gain of
two. 

\myfigure{graphics/pipe}{Architecture overview of the 7-bit Pipelined ADC with open-loop amplifiers}{pipe}

%It has been tradition to use switched capacitor circuits with
%high gain opamps to achieve the accurate gain of two. However, high
%gain opamps are difficult to design in nanoscale technologies, like
%the 65nm CMOS we choose for this design. The problem is the low output
%conductance of the transistors, and it is not uncommon with an
%intrinsic gain (gm/gds) of 10. For an accuracy of 7-bit in a closed
%loop switched capacitor circuit we would need
%a gain of around 46dB, which would require at least two stages (or
%gain boosting) in 65nm technology. 

One of the alternatives to the traditional operational amplifiers in MDACs is an open-loop amplifier,
like a common source amplifier, with a gain of two. This technique has
been used with success in a 12-bit pipelined ADC \cite{murmann03}
and 6-bit pipelined ADC \cite{Shen07}. Our design is based on
\cite{Shen07}. The MDAC architecture can be seen in
Figure \ref{p2fig:stage12}. Figure \ref{p2fig:stage12} shows stage one and stage
two. Stage one is in the amplifying phase, the SADC has made its
decision, and the control signals t0, t1 and t2 control transmission
gates that connect one of the two sampling capacitors to high
reference, common mode or low reference. The control signals t0, t1, t2
and the transmission gates implement the DAC. Stage two is shown
in the sampling phase. Here both capacitors are connected to the
input through transmission gates controlled by clock ip1. Each stage
needs three clock phases, p1, p1a and p2, where p1a is slightly advanced
over p1, and is used to sample the input when it is quiet. p1 and p2
are non-overlapping clock phases. Stage two uses ip1, ip1a and ip2, where
ip1=p2 and ip2=p1. All in all we need four clock phases for the
complete pipelined ADC, p1, p1a, p2 and p2a. The open-loop amplifier
is marked by x2 in Figure 
\ref{p2fig:stage12}. 
%The transmission gates
%and the SADC have been designed such that the number of clock signals
%is kept at a minimum, thus there are only four clocks signals distributed from
%the clock buffers. 


\myfigure{graphics/stage12}{ Stage 1 and stage 2 in the pipelined ADC}{stage12}

%------------------------------------------------------------------------
\subsection{Open-Loop Amplifier}
%------------------------------------------------------------------------

A detailed schematic of the open-loop amplifier can be seen in  Figure
\ref{p2fig:openamp}. It is a differential common source amplifier with
resistive load. The resistors are each $4k\Omega$, and we assume that they
will be calibrated at startup, so the resistor value is constant over
process. In this differential amplifier it is important that the common
mode variation is low. If not, the common mode voltage will drift as
the signal propagates through the pipelined stages, and may even turn later
stages off. To achieve low common mode variation we use a replica bias that
keeps the total current, $I_{tot} = I_{R1} +I_{R2}$, equal to two times
$I_{bias}$ over process, voltage and temperature variations. Through
this the common mode voltage is determined by 
%\eqn{
$V_{cm} = V_{dd} -  I_{tot}/2 \times (R1+R2)/2 $
%}
which keeps it constant over process and temperature variation. 

The
transistors M5/M6/M1/M2 are twice the size of M7/M8/M11/M12, as a
consequence $I_{tot} = 2I_{bias}$. We choose the common mode to be
0.6V, to get larger overdrive on the input transistors, and the swing
to be $\pm 0.2V$.

The gain of the common source amplifier, if we disregard the source
degeneration, is given by $A_o = g_{m1}R_1$
%
%\eqn{
%
%}
%where $g_{m1}$ is given by
%\eqn{
%g_{m1} = \frac{2I_d}{V_{gs} - V_{th}}
%}
%$g_{m1} = 2I_d/(V_{gs} - V_{th})$. 
%With $I_{tot}$ constant the transconductance varies over process due to
%changes in the threshold voltage, $V_{th}$. The variation in $V_{th}$ will change the gain
%of the amplifier. 
The gain will vary over process and temperature because of changes in
$g_m$, we compensate for some of this change by varying the vctrl
voltage of the source degeneration transistor, which  in effect
changes the effective source degeneration resistor and in turn changes
the gain of the amplifier. If we include the
source degeneration the gain
expression becomes
\eqn{
A_o = \frac{g_{m1}R_1}{1 + \frac{g_{m1}}{2 g_{ds13}}}
}

\myfigure{graphics/openamp}{ Open-Loop Amplifier}{openamp}

In \cite{Shen07} they used a replica MDAC stage configured in a feedback
loop to control the vctrl voltage such that the MDAC stage has a gain
of two. Since we only do simulation we have not included the replica
stage, the  vctrl voltage is  changed manually for each corner simulation. 

The gain of the MDAC is also affected by the input capacitance of the
amplifier as described in \cite{Shen07}.

Our main contribution to reduce the power of the pipelined ADC is the
transistors M3/M4, these turn off the the amplifier during sampling
phase when it is not needed. Because the drain and source nodes of
M3/M4 are low impedance, and the bias voltage, vb, stays constant,
the amplifier turns on quickly. 

In our design the bias current is  $I_{bias} = 100 \mu A$, which makes
the total current consumption of a MDAC stage $300 \mu A$, ergo the
power dissipation of five MDAC stages should be $1.5 mW$, assuming we
use a 1V supply. This is if we leave
the amplifiers on all the time. By turning off the amplifiers during sampling
the power dissipation is reduced to $P_{mdacs} = 5 \times 100 \mu A \times 1V+ \frac{1}{2}{5
  \times 200 \mu A} \times 1V = 1mW$, so we would expect an improvment
of 0.5mW when the amplifiers are turned off during sampling.

With this low current in the amplifiers, the input capacitance of the
next stage must be low. The sampling capacitors (C1, C2) are chosen at
50fF. The reason for choosing such a large value
compared to the required, is not capacitor matching concerns, which do not
come in to play at 7-bit, but rather the parasitic capacitance from
the amplifier input, which reduces the gain of the MDAC. We
compensate for some of this reduction in gain with the vctrl voltage.

Two other circuits dissipate power in the pipelined ADC, the SADCs and
the clock generator/buffers. The size of the clock buffers are mainly determined by
the load of the SADCs and routing capacitance. The power dissipation
in the SADCs are
determined by matching concerns of the comparators. The comparators used are the so
called Lewis-Gray
dynamic comparators introduced in \cite{cho95}. 

%------------------------------------------------------------------------
\subsection{Clock Generation}
%------------------------------------------------------------------------
Most
switched capacitor circuits use two non-overlapping clocks to control
the charge transfer, but since
we use bottom plate sampling there is an advanced phase 1 that transitions just
before phase 1. The advanced clock phase reduce the problem of input dependent charge injection
from the input switches. 

We use NMOS inputs in the comparators
due to the high common mode voltage, thus the comparators use a clock that
samples on the rising edge. To avoid distributing inverted clock
phases we invert the clocks of the transmission gates, which means that the
transmission gates are on when the clock signal is low, and off when
the clock signal is high. The clock buffers were scaled to drive 6 SADCs,
transmission gates, amplifier turn-off control signal and a 100fF
capacitance was added at the output of each clock buffer to model the the parasitic
capacitance added by metal routing.

%------------------------------------------------------------------------
\section{Results of Simulation}\label{p2results}
%------------------------------------------------------------------------
The ADC was simulated at transistor
level using Eldo from Mentor Graphics. A common mode of 0.6V, and a
swing of 0.2V, as previously mentioned, resulted in a differential
peak to peak of 0.8V with a 1V supply. Both the references and inputs are assumed to be
buffered off-chip, the buffers are not included in the simulation. Results of
Eldo simulations were post-processed in MATLAB, were an FFT was
performed and signal-to-noise-and-distortion (SNDR) was extracted.
In all simulations an input signal frequency close to the Nyquist frequency
was used, and an input signal amplitude of 0.8 times full scale range.

Mismatch simulations were performed in the typical corner at a
temperature of 27 degrees Celcius. A $2^7$ point FFT was used to estimate the
SNDR and 101 simulations were run to get an estimate of the standard
deviation of the SDNR due to mismatch. The mean SNDR was 40dB and the standard
deviation was 1.2dB. 

Four process corners (fast, slow, fast-slow, slow-fast) and three
temperature corners ($0^o, 27^o, 80^o$) were simulated. 
%Power supply
%voltage variation was also simulated, and it was found that the ADC
%breaks down at around 0.95V in slow corner down to speed issues. A positive shift
%in supply, to 1.1V, could be compensated with changing vctrl. 
When we vary vctrl to compensate for the threshold voltage changes, the
standard deviation in SDNR over process corners and temperature
corners is 2.35dB, with the worst corner being slow process and low
temperature. With constant vctrl the standard deviation increases to 3.45dB.

Figure \ref{p2fig:noise} shows an $2^{10}$ point FFT of the output from
a transient simulation with noise in the typical corner.  

\myfigure{graphics/siphon_fft}{A 1024 point FFT of the ADC output from a transient
  noise simulation. The harmonics of the fundamental are marked with diamonds.}{noise}

The total power dissipation for the simulated ADC in typical corner
(typical process, $27^o$ and 1V supply) is 2mW. If we leave the
amplifiers turned on during sampling it dissipates 2.6mW, with the increase being close
to the expected 0.5mW. Table
\ref{p2tab:performance} summarizes the achieved performance in the
typical corner.

The
Walden FOM of the ADC is 0.13pJ/step and the  Thermal FOM of
1.6fJ/step. 
The improvement is almost a factor four compared to the Thermal FOM of
5.7fJ/step from \cite{Chen06}, bearing in mind that they have proven
silicon, and that the difference might be eaten up by layout
parasitics. In addition, we operate at a lower speed and a higher SNDR,
which makes it more straightforward to achieve a good figure of merit. However, \cite{Chen06} was a SAR while our ADC is
a pipelined ADC, making the point the two architectures can compete on
equal grounds with respect to figure of merit at reasonable speeds.

%The Thermal FOM of our ADC is still two orders of magnitude worse than
%the apparent ``thermal limit'' of 0.01fJ/step. 
%Much of this can be
%explained by small, but still the relatively large sampling
%capacitors of 50fF, if we compare to the required sampling capacitor
%of around 6fF. Even in a state of the art technology, like the 65nm
%CMOS, it is difficult to design a pipelined ADC
%which is limited by thermal noise.  

 
\begin{table}[ht]
\centering 
\caption{Preformance summary of the 7-bit Pipelined ADC}
\begin{tabular}{l|r}
\label{p2tab:performance}
Technology & 65nm LP CMOS\\
Input Voltage Peak-to-Peak & 0.8V\\
Supply Voltage & 1V \\
Sampling Frequency & 200MHz\\
SNDR & 40dB \\
ENOB & 6.3dB \\
Power Dissipation & 2mW \\
Walden FOM & 0.13pJ/step\\
Thermal FOM & 1.6fJ/step\\
\end{tabular}

\end{table}

\section{Future Work}
Low resolution pipelined ADCs with open-loop amplifiers make for an
interesting architecture, it may well be the most efficient, straightforward way
to implement high speed low resolution ADCs at the present time. 
More work is needed to try to reduce the sampling capacitance,
and we believe that cutting the parasitic capacitances down, through
architecture or technology changes, is the most effective way
to increase effectiveness of high speed low resolution ADCs.


\section{Conclusion}
We presented the design of a
7-bit 200MS/s pipelined ADC with switched open-loop amplifiers in a
65nm CMOS technology. As a
result of turning
off the open-loop amplifiers during sampling the power dissipation was
reduced by 23\%. The ADC achieved a SNDR
of 40dB close to the Nyquist frequency, with a power dissipation of
2mW, which resulted in a 
Waldon FOM of 0.13pJ/step and a Thermal FOM of 1.6fJ/step.


%------------------------------------------------------------------------
\section*{Acknowledgments}
%------------------------------------------------------------------------
Financial support from the Norwegian Research Council through the
project Smart Microsystems for Diagnostic Imaging in Medicine (project
number 159559/130) and the project ASICs for Microsystems (project
number 133952/420) is gratefully acknowledged.

%-------------------------------------------------------
%Figures
%-------------------------------------------------------


%\bibliography{IEEEabrv,/home/wulff/svnwork/wulff/work/ntnu/phd/phd,/home/wulff/svnwork/wulff/work/ntnu/phd/IEEE_J_JSSC.bib}


%\end{document}
