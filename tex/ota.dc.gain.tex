
%#####################################################################
\chapter{Opamp DC gain}
%#####################################################################
One of the challenges with opamp based switched-capacitor circuits in
nano-scale CMOS is the DC gain of the opamp. High accuracy Nyquist SC ADCs
need high DC gain.

How difficult it is to design an opamp in nano-scale CMOS is set by
the intrinsic gain of the transistors ($g_m/g_{ds}$). If we restrict
our investigations to
three types of opamps--- differential pair, folded cascode, and folded
cascode with gain boosting ---, and different permutations of these,
we can give an estimate of the DC gain based on the
intrinsic gain  of a technology. 

\subsubsection{Common Source (CS)}
In a differential pair the amplifier is a common source amplifier.
The DC gain of a common source amplifier can be approximated by
\eqn{
  A_{CS} = \frac{1}{2} G_i
}


\subsubsection{Folded Cascode (FC)}
The FC is common in switched
capacitor circuits. There are many variations on the FC, but the basic circuit
can be seen in Figure \label{fig:foldedcasc}. The input differential
pair provide the transconductance, and the cascode  increase the
output resistance. If we assume that the transconductance of the input
transistors are equal to that of the cascode transconductance, and that
the intrinsic gain is defined as $G_i = g_m/g_{ds}$, we can
approximate the DC gain by
\eqn{
  A_{FC} = g_{mi} \times g_{m} r_{ds}^2/2 = \frac{1}{2} G_i^2
}


\paragraph{Two Stage OTA}
If we make a two-stage amplifier the total DC gain is a multiplication of
the individual gains. For example a folded cascode with a common
source output stage has a gain of
\eqn{
A_{tot} = A_{FC}A_{CS} = \frac{1}{4}G_i^3
}

\paragraph{Gain Boosting}
Gain boosting is common in folded cascode OTA to further increase the
output resistance. The technique increase the output
resistance proportional to the gain of the gain boosting
amplifier. Ergo the gain of a folded cascode with folded cascode gain
boosting is
\eqn{
A_{FC[FC]} =
A_{FC}A_{FC}
= \frac{1}{4} G_i^4
}

Table \label{tab:otagain} is an overview of the estimated DC gain for
different configurations of OTA. The first four OTAs  are well seasoned
OTAs and can be found in many publications \cite{berntsen05}, \cite{ahmed05}. The last
two are not trivial to design.

\begin{table}[ht]
\caption{Gain estimates for different OTA architectures}
\centering 
\begin{tabular}{ l | l | c }

\label{tab:otagain}
Gain Name&Architecture & Gain Estimate\\
\hline 
$A_{CS}$&CS & $\propto G_i$\\\hline 
$A_{FC}$&FC & $\propto G_i^2$\\\hline 
$A_{FC-CS}$&FC with CS second stage & $\propto G_i^3$\\\hline 
$A_{FC[FC]}$&FC with FC gain boosters & $\propto G_i^4$\\\hline 
$A_{FC[FC]-CS}$&FC with FC gain boosters and CS second stage & $\propto G_i^5$\\\hline 
$A_{FC[FC]-FC}$&FC with FC gain boosters and FC second stage & $\propto G_i^6$\\
\end{tabular}
\end{table}


\subsection{Opamp gain in different technologies}

In the recent CMOS technology nodes we have seen a drastic reduction in
intrinsic gain, making it harder and harder to design high DC gain
opamps. In 65nm we have reached a point were it might not be practical
to design opamps with 100dB gain. Such an opamp will be a challenge
to design. Some of the challenges in the design of
such an opamp include ensuring stability, achieving low power
dissipation and designing multiple stable common mode
feedback loops.

The intrinsic gain can be used to estimate the DC gain of
opamps. Table \label{tab:ingain} shows the intrinsic gain for
transistors in different technology nodes. We've used a width to
length ratio of ten and a length of 1.2 times the minimum length. All
transistors were biased at an effective gate overdrive of $V_{EFF} = V_{GS} - V_{TH} =
V_{DD}/8$ and a drain to source voltage of $V_{DS} =
V_{DD}/2$. Table \label{tab:ingain} shows the gain estimates in dB for different
OTA architectures. In 350nm and 180nm  more than $100dB$ gain is feasible without
too much difficulty. But in 90nm and 65nm it would be very challenging
to design an opamp with more than $100dB$ gain. The
values in Table \label{tab:ingain} are what you get for free in each
technology, any more gain for a given topology will cost time and
effort.

% Table \label{tab:realota} shows the gains of
%some published OTAs.

\begin{table}[ht]
\caption{OTA gain estimates for different technology nodes and OTA architectures. Simulations in SPICE using $W/L=10, V_{EFF}=V_{DD}/8, V_{DS}
  = V_{DD}/2, L=1.2L_{min}$. All gains are in dB.}
\centering 
\begin{tabular}{ l | c | c | c | c | c | c | c }

\label{tab:ingain}
Tech. node&$G_i$&$A_{CS}$&$A_{FC}$&$A_{FC-CS}$&$A_{FC[FC]}$&$A_{FC[FC]-CS}$&$A_{FC[FC]-FC}$\\
\hline 
65nm	&9	&13     &32	&45	&64	&77	&96\\
90nm	&13	&16     &39	&55	&77	&93	&116\\
130nm	&17	&19     &43	&62	&86	&105	&130\\
180nm	&26	&22     &51	&73	&101	&123	&152\\
350nm	&60	&30     &65	&95	&130	&160	&195\\
\end{tabular}

\end{table}
