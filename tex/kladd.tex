



%---------------------------------------------------------------------
\section{Least Significant Bit}
%---------------------------------------------------------------------
The Least Significant Bit (LSB) is the minimum step we have in the
ADC. We define $V_{LSB}$ as as 
\eqn{
	V_{LSB} = \frac{V_{pp}}{2^N}
}
where $V_{pp}$ is the peak to peak signal swing of the ADC and N is the
number of bits.

\begin{exmp}
Using $V_{pp}$ = 0.4V and N = 10 we get
\eqn{
	V_{LSB} = \frac{0.4V}{1024} = 0.39mV 
}

\end{exmp}

%---------------------------------------------------------------------
\section{Stage Capacitance}
%---------------------------------------------------------------------

Figure \reg{stagecaps} shows the capacitors connected to the
current source during charge transfer. The capacitance $C_{01}$ and
$C_{02}$ are given by
\eqna{
  C_{01} &{}={}& C_{1s1} + C_{ps1}\\
  C_{02} &{}={}& \frac{C_{2s1}C_{01}}{C_{2s1} + C_{01}}
} 
, where $C_{1s1}$ is the capacitor one in the first stage, $C_{ps1}$
is the parasitic capacitor of the comparator in the first stage and
$C_{2s1}$ is the second capacitor in the first stage. The total
capacitance is
\eqn{
  C_{tot} = \frac{C_{2s1}(C_{1s1} + C_{ps1})}{C_{2s1} + C_{1s1} + C_{ps1}} + C_{ss2}
}
, where $C_{ss2}$ is the sampling capacitor in the second stage.
\myfigure{graphics/caps}{Capacitance network during charge
  transfer}{stagecaps}

%---------------------------------------------------------------------
\subsection{Thermal Noise Limited Capacitance}
%---------------------------------------------------------------------
We assume that the capacitances are limited by thermal noise.
The 
average noise power of the quantization noise is given by
\eqn{
\label{eq:quantnoisepower}
 \overline{ V_q^2(t)} = \frac{LSB^2}{12}
}
and thermal noise power is given by
\eqn{
\label{eq:thermalnoisepower}
  \overline{V^2(t)} = \eta_t \frac{kT}{C}
}
where $\eta_t$ is a factor larger than one which represents how much
of the noise budget should be attributed to thermal noise.
Setting \req{quantnoisepower} equal to  \req{thermalnoisepower} we get
\eqn{
\label{eq:capsize2}
C = \frac{\eta_t 12kT}{LSB^2} = \frac{\eta_t 12kT\times 2^{2N}}{V_{pp}^2}
}

\begin{exmp}

Using \req{capsize2} with $LSB = 0.39mV$ and $\eta_t=1.5$, the
sampling capacitor, $C_{ss1}$, has to be
\eqn{
C_{ss1} > \frac{1.5 \times 12 \times 1.38\times10^{-23} J/K \times 353 K }{  (0.39mV)^2}
= 3.84\times10^{-13} J/V^2 \approx 600fF
}
For the second stage $LSB = 2 \times 0.39mV$ so the capacitance
$C_{ss2}$ has to be
\eqn{
 C_{ss2} >\frac{1.5\times 3 \times 1.38\times10^{-23} J/K \times 353 K }{  (0.39mV)^2}
= 9.61\times10^{-14} J/V^2 \approx 200fF
}
Which means that $C_{1s1} = C_{2s1} = 300fF$ and the total capacitance
at the output, if we ignore parasitics, is given by
\eqn{
 C_{tot} = \frac{300fF \times 300fF}{300fF + 300fF} + 200fF = 350fF
}
If we allow for some parasitics we get a total output capacitance of
around $400fF$

\end{exmp}


%---------------------------------------------------------------------
\subsection{Virtual Ground Voltage Change}
%---------------------------------------------------------------------
Figure \reg{stagecaps} showed the capacitances in the stage during the
multiply phase. If we ignore the parasitic capacitance $C_p$, the
current in the two brances are equal. We can therfore write
\eqna{
C_2(\Delta V_O - \Delta V_{VG}) &{}={}& C_1 \Delta V_{VG}\\
C_2 \Delta V_O &{}={}& (C_1  + C_2) \Delta V_{VG} 
}
and with $C_1 = C_2$ a change in $V_O$ causes a change in $V_{VG}$ of
\eqn{
\label{eq:cbscvgchange}
\Delta V_{VG} = \frac{\Delta V_O}{2}
}

%---------------------------------------------------------------------
\section{Integration Current}
%---------------------------------------------------------------------
The parameters of integration current is determined by the need to
charge the output capacitance to a certain voltage and the accuracy
needed at th output. First we will derive epressions for the necessary
integration current as a function of speed. Next we will derive
expressions for the neccessary output resistance of the current
source.

%---------------------------------------------------------------------
\subsection{Neccessary Integration Current}
%---------------------------------------------------------------------

To charge the output capacitance within half the sampling period the
output current  is given by
\eqn{
	I_1 = C_{tot} \frac{dV_o}{dt} = C_{tot} \frac{V_{max}}{T_s/2 - T_R} 
}
where $T_s = 1/F_s$ and $F_s$ is the sampling frequency, $T_R$ is the
time needed to reset the output and a comparator delay.

\begin{exmp} Using $C_{tot} = 400fF$, $T_R = 1ns$, $T_s/2 = 5ns$ and $V_{max} = 0.8V$  
\eqn{
	I_1 = 400fF \times \frac{0.8V}{4ns} = 80 \mu A
}

\end{exmp}

%---------------------------------------------------------------------
\subsection{Necessary Output Resistance}
%---------------------------------------------------------------------
The output resistance needed depends on the comparator delay and the
integration current. A finite output resistance will lead to a gain
error in the stage which must be less than 1 LSB if gain calibration
is not used. 

\eqna{
	\Delta V &{}={}& SR \times T_C \nonumber\\
	2 \eta_g \frac{V_{pp}}{2^N} &{}={}& \frac{V_{pp}}{r_{ds}\times C_{tot}} \times
T_C \nonumber \\
	r_{ds} &{}={}& \frac{2^{N-1}\times T_C}{\eta_g \times C_{tot}}
}
where $\eta_g$ is the portion of the error budget allotted to gain
errors, $V_{pp}$ is the peak to peak signal swing, $T_C$ is the
comparator delay and $N$ is the number of bits.

\begin{exmp} Using $\eta_g = 0.5$, $N=10$, $C_{tot} = 400fF$ and $T_C
= 300ps$
\eqn{
	r_{ds} =\frac{2^9\times300ps}{0.5\times 400fF} = 768 k\Omega
}

\end{exmp}
which should not be too hard using wide swing enhanced cascode current sources.
